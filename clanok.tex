% Metódy inžinierskej práce

\documentclass[10pt,twoside,slovak,a4paper]{coursepaper}

\usepackage[slovak]{babel}
%\usepackage[T1]{fontenc}
\usepackage[IL2]{fontenc} % lepšia sadzba písmena Ľ než v T1
\usepackage[utf8]{inputenc}
\usepackage{graphicx}
\usepackage{url} % príkaz \url na formátovanie URL
\usepackage{hyperref} % odkazy v texte budú aktívne (pri niektorých triedach dokumentov spôsobuje posun textu)

\usepackage{cite}
%\usepackage{times}

\pagestyle{headings}

\title{Názov\thanks{Semestrálny projekt v predmete Metódy inžinierskej práce, ak. rok 2034/24, vedenie: Ing. Vladimír Mlynárovič PhD.}} % meno a priezvisko vyučujúceho na cvičeniach

\author{Tomáš Majerčík\\[2pt]
	{\small Slovenská technická univerzita v Bratislave}\\
	{\small Fakulta informatiky a informačných technológií}\\
	{\small \texttt{xmajercik@stuba.sk}}
	}

\date{\small 30. september 2023} % upravte



\begin{document}



\maketitle

\begin{abstract}

V tejto práci sa zaoberáme porovnaním softvérov Chrome, Firefox, Edge a Brave. Pozrieme sa na ne nie len z hľadiska bezpečnosti ale aj z pohľadu ako tieto prehliadače zobrazujú a získavajú informácie, ktorý prehliadač najmenej zaťažuje hardware (CPU (procesor) alebo RAM (operačná pamäť)), ktorý je najpopulárnejší, najdostupnejší a najobľúbenejší medzi užívateľmi, ktorý ponúka najlepšie prostredie a prostriedky pre vývoj webových aplikácii a samozrejme aj pohľad na poskytnutú bezpečnosť či súkromie. V článku sa snažíme na základe získaných informácii o porovnanie výhod a nevýhod daných prehliadačov. Výsledky tejto práce môžu slúžiť ako prostriedok pre čitateľov pri vyberaní ich nového obľúbeného softvéru.


\end{abstract}



\section{Úvod}









\section{Záver} \label{zaver} % prípadne iný variant názvu

%\cite{literature}

%\acknowledgement{Ak niekomu chcete poďakovať\ldots}


% týmto sa generuje zoznam literatúry z obsahu súboru literatura.bib podľa toho, na čo sa v článku odkazujete
\bibliography{literatura}
\bibliographystyle{abbrv} % prípadne alpha, abbrv alebo hociktorý iný
\end{document}
